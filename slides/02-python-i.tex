\input{slheader}
\title[Python I]{Introduction to Python Programming}
\begin{document}

\frame{\maketitle}

\note{
    Goals for this session:

    - Be able to run a program.
    - Be able to run interactively.

    - Rough understanding of name/object concept.
    - Understand control flow.
}

\begin{frame}[fragile]
\frametitle{Python}

Let's digress for a moment discussing the language\ldots
\end{frame}


\begin{frame}[fragile]
\frametitle{Python Language History}

\begin{block}{History}
\begin{itemize}
\item Python was started in the late 80's.
\item It was intended to be both \alert{easy to teach} and \alert{industrial strength}.
\item It is (has always been) open-source.
\item In the last 10 years, it has become one of the most widely used languages (top 10).
\end{itemize}
\end{block}
\end{frame}

\begin{frame}
\frametitle{Popularity}

\centering
\includegraphics[width=.8\textwidth]{language-ranking-0912.png}

\end{frame}

\begin{frame}
\frametitle{Popularity}

\centering
\includegraphics[width=.8\textwidth]{language-ranking-0912-zoom.png}

\end{frame}
\begin{frame}
\frametitle{Python Versions}

\begin{block}{Python Versions}
\begin{itemize}
\item The current versions of Python are \alert{2.7} and \alert{3.3}
\item This class assumes you have 2.6--2.7
\item There are some small differences when compared to version 3.x
\end{itemize}
\end{block}

\end{frame}


\begin{frame}[fragile]
\frametitle{What is a Computer?}

\begin{enumerate}
\item Memory
\item Processor
\item Magic
\end{enumerate}
\end{frame}

\begin{frame}[fragile]
\frametitle{Python Model}

\begin{enumerate}
\item Objects
\item Operations on objects
\item Magic
\end{enumerate}
\end{frame}


\begin{frame}[fragile]
\frametitle{Python Example}

\begin{python}
print "Hello World"
\end{python}
\end{frame}


\begin{frame}[fragile]

\begin{block}{Running Python}
\begin{enumerate}
\item From a file
\item Interactively
\end{enumerate}
\end{block}

\end{frame}

\begin{frame}[fragile]
\frametitle{Computer Program}

\begin{block}{helloword.py}
\begin{python}
print 'Hello World'
\end{python}
\end{block}
\end{frame}

\begin{frame}[fragile]
\frametitle{Running a Program}
\begin{enumerate}
\item Shell
\item IDE
\end{enumerate}
\end{frame}

\begin{frame}[fragile]

\bigskip
\bigskip
\bigskip
Let me show you a demonstration\ldots

\note{
\begin{enumerate}
\item Demo shell with vim
\item Demo one IDE
\end{enumerate}
}


\end{frame}

\begin{frame}[fragile]
\frametitle{More Complex Example}

What is 25 times 5?

\pause
\begin{python}
print 25 * 5
\end{python}
\note{Demo the python shell.

This is basically a glorified calcular}
\end{frame}

\begin{frame}[fragile]
\frametitle{More Complex Example}

\begin{python}
name = 2
other = 3
yetanother = name + other
name = 5
print yetanother + name
\end{python}
\end{frame}

\begin{frame}[fragile]
\frametitle{Blackboard demonstration}

\note{Use the blackboard to introduce the idea of objects, values and names.}
\end{frame}

\begin{frame}[fragile]
\frametitle{Conditionals}

\begin{python}

if <condition>:
    <statement 1>
    <statement 2>
else:
    <statement 3>

\end{python}
\end{frame}


\begin{frame}[fragile]
\frametitle{Example}

\begin{python}
x = 2.03
v = -2.523 + 0.22 * x  + .51 * x**2

if v > 0:
    print 'positive'
else:
    print 'negative'
\end{python}

\end{frame}

\begin{frame}[fragile]
\frametitle{Example}

\begin{python}
x = 2.03
v = -2.523 + 0.22 * x  + .51 * x**2

if v > 0:
    print 'positive'
elif v == 0:
    print 'zero'
else:
    print 'negative'
\end{python}

\end{frame}

\end{document}
