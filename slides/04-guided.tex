\documentclass{beamer}
\usetheme{IMM}

\usepackage{pgf,pgfarrows,pgfnodes,pgfautomata,pgfheaps,pgfshade}
\usepackage{amsmath,amssymb}
\usepackage{colortbl}
\usepackage[english]{babel}
\usepackage{booktabs}
\usepackage{slpython}
\usepackage{underscore}
\usepackage{bm}

\author{Luis Pedro Coelho}
\institute{Programming for Scientists}

\graphicspath{{../figures/}{../figures/generated/}{../images/}}

\newcommand*{\code}[1]{\textsl{#1}}
\newcommand*{\Reals}[1]{R}
\newcommand*{\Assign}{\ensuremath{\leftarrow}}
\newcommand{\creditto}[1]{%
\begin{flushright}
(#1)
\end{flushright}%
}

\title{Guided Exercises}
\begin{document}
\frame{\maketitle}

\begin{frame}
\frametitle{Goals for this hour}

\begin{itemize}
\item A quiz
\item Do a few exercises.
\item Play around.
\item You can work alone, in pairs, in triples,\ldots
\end{itemize}

\end{frame}

\begin{frame}[fragile]
\frametitle{What Happens Here?}
\begin{python}
x = 2
v = x*x
v = x + 1
if x > 2:
    print x
else:
    print v
\end{python}

\begin{enumerate}
\item Prints \alert{2}
\item Prints \alert{3}
\item Prints \alert{4}
\item Something else
\end{enumerate}

\end{frame}

\begin{frame}[fragile]
\frametitle{What Happens Here?}
\begin{python}
x = 2
for i in range(4):
    x = x + 1
print x
\end{python}

\begin{enumerate}
\item Prints \alert{4}
\item Prints \alert{6}
\item Prints \alert{3}, \alert{4}, \alert{5}, \alert{6}
\item Prints \alert{3}, \alert{4}, \alert{5}
\end{enumerate}

\end{frame}

\begin{frame}[fragile]
\frametitle{What Happens Here?}

\begin{python}
x = 10

if x > 5:
    print 'greater than 5'
elif x > 3:
    print 'greater than 3'
else:
    print 'not that big, really'
\end{python}

\end{frame}

\begin{frame}[fragile]
\frametitle{How Many Above 35?}

\begin{python}
ages = [10, 54, 23, 90, 23, 35]
\end{python}

How many are above 35?

\pause

\begin{python}
above35 = 0
for v in ages:
    if v > 35:
        above35 += 1
print above35
\end{python}

\end{frame}


\begin{frame}[fragile]
\frametitle{What does this print?}

\begin{python}
ages = [10, 54, 23, 90, 23, 35]
vi = 0
while ages[vi] <= 35:
    vi += 1
print vi
\end{python}

Prints

\begin{enumerate}
\item 0
\item 1
\item 2
\item 54
\end{enumerate}

\end{frame}


\begin{frame}[fragile]
\frametitle{What happens here?}

\begin{python}
ages = [10, 54, 23, 90, 23, 35]
vi = 0
while ages[vi] <= 100:
    vi += 1
print vi
\end{python}

\end{frame}


\begin{frame}
\frametitle{Lists I}

How do you access the first element of a list?

Assume \texttt{list} is a list:

\begin{enumerate}
\item \texttt{list[1]}
\item \texttt{list[0]}
\item \texttt{list[-1]}
\item \texttt{list(0)}
\item \texttt{list(-1)}
\item \texttt{list(1)}
\end{enumerate}
\end{frame}

\begin{frame}
\frametitle{Lists II}

How do you access the last element of a list?

Assume \texttt{list} is a list:

\begin{enumerate}
\item \texttt{list[1]}
\item \texttt{list(-0)}
\item \texttt{list[-1]}
\item \texttt{list(-1)}
\item \texttt{list(1)}
\item \texttt{list[-0]}
\end{enumerate}
\end{frame}

\begin{frame}
{}

\bigskip
\bigskip
\bigskip
Exercises
\end{frame}

\begin{frame}[fragile]
\frametitle{Object Identity}

What is the difference between the following two code examples:

A)
\begin{python}
A = [1, 2, 3]
B = [1, 2, 3]
\end{python}

B)

\begin{python}
A = [1, 2, 3]
B = A
\end{python}

Write a small piece of code (should be 2 or 3 lines) that behaves differently
if you insert it after each of the two segments above.

\pause

\begin{python}
B[0] = 0
print A
\end{python}

\end{frame}

\end{document}
